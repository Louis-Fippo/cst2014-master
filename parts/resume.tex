
\chapter{Résumé}

L'étude et la compréhension des systèmes biologiques repose sur notre capacité à construire des modèles de ces systèmes afin d'en étudier  
les propriétés. Le fonctionnement des systèmes biologiques peut être modélisé  par des grands réseaux d'interactions composés de gènes,
protéines et complexes qui interagissent entre eux pour produire des dynamiques. Grâce à la disponibilité des données d'expressions des gènes,
nous pouvons désormais observer et caractériser les dynamiques des composants des réseaux de régulations géniques. Le but de ce travail est de pouvoir 
générer de façon automatique des modèles formels des réseaux de régulations géniques en y intégrant des données temporelles mesurées pendant des expériences.
L'idée est d'aller un peu loin dans la prise en compte de l'aspect temporel et stochastique du fonctionnement des composants biologiques dans les modèles 
formels d'analyse. Ce travail essaye d'apporter plusieurs éléments de contributions dans ce sens. Nous proposons tout d'abord une façon d'estimer les paramètres temporels
et stochastiques des séries temporelles pour la simulation d'un modèle qui les explique. Puis nous proposons une génération semi-automatique des réseaux biologiques en  modèle en frappes de processus qui intègre 
les données temporelles et stochastiques estimées. Enfin nous effectuons  une discretisation des données des séries temporelles pour permettre une comparaison avec les 
résultats issus des simulations des modèles.




%De nombreuses approches de modélisation ont été proposé et leurs expressivités et leurs complexités sont fortements corrélés 
%et varient  en fonction de ce que nous souhaitons capturer de la réalité du phénomène que nous modélisons. 