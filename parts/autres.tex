\chapter{Autres activités}

\section{Production scientifique}

\textbf{Article court soumis et accepté avec présentation d'un exposé oral}\\
Louis Fippo Fitime, Andrea Beica, Olivier Roux and Carito Guziolowski:
\textbf{Integrating time-series data on large-scale cell-based models: application to skin differentiation}.
Proceedings of the Evry Spring School, pages 57-72. 2014.

\section{Ecoles et Formations suivies}

\begin{itemize}
\item \textbf{AVANCÉES EN BIOLOGIE DES SYSTÈMES ET DE SYNTHÈSE (ASSB'14)},
Modélisation de systèmes biologiques complexes dans le contexte de la génomique.\\
École Thématique de Recherche\\
Du 24 au 28 mars 2014, Évry, France 

\item \textbf{École d'été sur la Modélisation et Vérification des Systèmes Parallèles (MOVEP'14)},\\
7-11 juillet 2014, Nantes, France 

\item \textbf{Formation doctorale : Estimation des incertitudes}\\
Type formation Cours \\
Durée: 15:00 heures
\end{itemize}

\section{Enseignements}

%\bigskip

\noindent
\textbf{Méthodes logicielles (MELOG) :} 2\textsuperscript{e} année (semestre 6)\\
Programmation orientée objet, structures de données et langage Java\\
\textbf{Responsable :} Guillaume MOREAU
\begin{itemize}
  \item 2 groupes de TP, soit 28 heures
\end{itemize}

\noindent
\textbf{Algorithmique et programmation (ALGPR) :} 1\textsuperscript{ère} année (semestre 6 et 7)\\
Introduction à l'algorithmique et applications au langage C\\
\textbf{Responsable :} Vincent TOURRE
\begin{itemize}
  \item 2 groupes de TP, soit 40 heures
\end{itemize}

\section{Activités collectives}
Membre du bureau en tant que vice-trésorier et correspondant IRCCyN des membres de l'\emph{Association des Étudiants en Doctorat de l'École Centrale de Nantes} (AED).
