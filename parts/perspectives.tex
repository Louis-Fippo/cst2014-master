\chapter{Perspectives}

A court terme, nous allons proposer des simulations du modèle complet. Pour ce faire nous devons mettre ensemble les sous-graphes pour reconstituer le modèle
complet. Cette reconstitution exige la msie en oeuvre du mécanisme de synchronisation entre les composants qui ont une influence indépendante sur un autre 
composant. En effet, nous avons observé que des composants qui agissent de façon indépendante sur un autre composant peuvent générer des oscillations. Ce comportement
n'est pas cohérent avec la dynamique que voulons générer. Pour cette réaison nous avons décidé de mettre en oeuvre le mécanisme de synchronisation dans le formalisme 
des frappes de processus. En plus de cette perspective immédiate, nous allons présenter dans la suite les autres perspectives de ce travail.

\section{Notion de reconnaissance automatique  de trace}\label{sec:trace}
On peut comparer deux systèmes de transitions pour voir leur évolution au fur et à mesure que les transistions sont tirées. L'évolution d'un système peut être 
représentée par la trace du chemin issu de l'état initial ou un autre. On peut alors analyser les traces et voir comment un système se comporte par rapport à un autre.
En particulier, on peut chercher à savoir si des traces sont identiques, si l'une est incluse dans l'autre, si les deux traces ont les sous-traces communes spécifiques.
Nous comptons sur la base de cette idée mettre en oeuvre un mécansme de reconnaissance automatique des traces qui seront générées après simulation du modèle. L'objectif est
de pouvoir comparer de façon automatique un résultat d'une simulation avec une courbe discrete sur la base des critères objectifs.


\section{Etude stochastique}\label{sec:verification}

Le modèle que nous avons construis est un modèle qui prend en compte l'aspect stochastique des systèmes biologiques. Nous nous proposons d'effectuer une étude stochastique de 
ce modèle afin de pouvoir mettre en évidence un ensemble de propriétés liées à l'aspect stochastique du focntionnement du modèle. Un des objectifs majeur de cette perspective est 
de pouvoir faire des prédictions stochastiques des dynamiques que nous pourrions observer en fonction du nombre de simulation et des paramètres stochastiques du modèle. Nous 
espérons aussi pouvoir mettre en évidence un ensemble de caractéristiques stochastiques qui vont nous permettre d'influencer la dynamique du modèle.

\section{Vérification logique et Processus Clés} \label{sec:keyplayer}
Outre l'étude stochastique du modèle, nous allons nous intéresser à une vérification logique des propriétés du modèle afin de proposer les éléments pour le contrôle de 
la dynamique du réseau. Nous comptons nous intéresser à la détection des processus clés. Ici un processus clé est un processus tel que son activité(actif ou inactif) modifie la dynamique du réseau ou 
d'une sous partie du réseau.

\section{Mise en oeuvre sur un cas réel biologique}\label{sec:miseenoeuvre}
Les résultats concernants l'analyse dynamique du système et les processus clés seront validés expérimentalement par des analyses fonctionnelles
des gènes clés du système de différentiation cellulaire via l'interférence basée sur shRNA dans les kératinocytes humains primaires.

